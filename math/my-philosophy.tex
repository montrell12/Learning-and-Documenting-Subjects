\documentclass[12pt]{article}
\usepackage[margin=1in]{geometry}
\usepackage{parskip}

\title{\textbf{My Philosophy on Mathematics}}
\author{Montrell Pegues}

\begin{document}

\maketitle

\section*{Foundations of Mathematics and Axioms}

To me, to really get math is to accept it's based on observations about the axioms or truths of numbers and the patterns they form, the world, and the hypothetical world that if it were to exist we would know its properties and functions.

Math is not built on a solid foundation where we know exactly why the axioms are true. For example, I couldn't tell you why the universe gave the pattern to triangles that the product of the adjacent and acute sides of a triangle will always equal the hypotenuse.

Math's foundation starts on the clouds where we accept the axioms as true and build from there. We are forever trying to get that foundation to the ground while simultaneously building higher and higher into the sky.

\section*{Numbers and their Functions}
When I think about numbers, I see them as identities attached to a specific thing (which means anything physical or hypothetical), numbers correlated to a pattern formed from those numbers, also causes of a pattern forming from a specific thing or things.

The identities of those things give us results to the relationship formed, which we can also look at as identities when we track the results from different inputs on those relationships.

A good way to understand why numbers came to be is to imagine yourself, or our ancestors as hunter-gatherers, in a time when the modern number system isn't created yet. You're gathering apples for your village and you know\ldots

When our ancestors tried to group people in a formation the only way was to singley guide a person and point to where they will be

\section*{Mathematics as a Language and The Reason Why the Brain Naturally Doesn't Use Numbers for Pattern Recognition}

\section*{Math Symbols And finding Patterns and simplifing Equations}
Math symbols are just a creation from a certain pattern formed in numbers or set of numbers to make dealing with equations easier. i could have the equation 2+2+2+2+5 but i see a lot of reduant 2's being added together so instead i could condense it down with multipulcation so i know i have 2 as the base and its being added 4 times so i can do 2x4+5 to simplify the equationn 
  

\section*{Equations}

\section*{Patterns and their Formations}

\section*{Building and Solving Equations from Axioms}

\end{document}

